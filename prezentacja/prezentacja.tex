\documentclass{beamer}
%
\mode<presentation>
{
  \usetheme{CambridgeUS}

  \setbeamercovered{transparent}
}
\usefonttheme[onlylarge]{structurebold}
\usepackage[polish]{babel}
\usepackage[utf8]{inputenc}
%\setbeameroption{show notes}
%
%\usepackage{times}
%\usepackage[T1]{fontenc}
\usepackage{ae}
\usepackage{tikz}
\usepackage{ulem}
%
\newcommand{\marked}[1]{{\bf #1}}
%
\title{Tagery morfosyntaktyczne dla języka polskiego}
%
\author{Łukasz Kobyliński \and Witold Kieraś}
%
\institute[IPI PAN]{%
     Instytut Podstaw Informatyki Polskiej Akademii Nauk\\
     ul. Jana Kazimierza 5, 01-248 Warszawa, Poland}
%
\date{7.12.2015}
%
\setbeamercolor{block title}{use=structure,fg=white,bg=purple!75!black}
\setbeamercolor{block body}{use=structure,fg=black,bg=white!20!white}
%
\setbeamertemplate{section page}
{
    \begin{centering}
    \begin{beamercolorbox}[sep=12pt,center]{part title}
    \usebeamerfont{section title}\insertsection\par
    \end{beamercolorbox}
    \end{centering}
}
%
\begin{document}
\begin{frame}
  \titlepage
\end{frame}

\begin{frame}
\frametitle{Plan}
\tableofcontents
\end{frame}

\section{Wprowadzenie}
% problem
% projekt
% plany
\begin{frame}{Wprowadzenie}
\structure{Cel prezentacji}
\end{frame}

\section{Tagery języka polskiego -- przegląd rozwiązań}
\frame{\sectionpage}

% analiza
% typowe problemy
\begin{frame}{Tagery morfostynaktyczne dla języka polskiego}
\structure{Tagery uwzględniające tagset NKJP}
\begin{itemize}
\item Pantera [Acedański 2010] -- adaptacja algorytmu Brilla do języków bogatych morfologicznie, takich jak polski,
\item WMBT [Radziszewski and Śniatowski 2011] -- tager oparty na uczeniu pamięciowym, rozbudowany o wielowarstwowość dla uwzględnienia wielu atrybutów znakowania w języku polskim,
\item Concraft [Waszczuk 2012] -- tager warstwowy, oparty na Conditional Random Fields (CRF); wyniki dezambiguacji morfosyntaktycznej przekazywane są z jednej warstwy do drugiej,
\item WCRFT [Radziszewski 2013] -- również oparty na CRF; osobne modele wykorzystywane są do dezambiguacji poszczególnych atrybutów opisu morfosyntaktycznego.
\end{itemize}
\end{frame}

\section{Tagery języka polskiego -- analiza ilościowa}
\frame{\sectionpage}

\begin{frame}{Metoda ewaluacji}
\structure{Miara jakości znakowania}
\begin{itemize}
\item ze względu na możliwość wystąpienia różnic w segmentacji pomiędzy wynikiem znakowania, a złotym standardem, wykorzystujemy dolne ograniczenie trafności (\emph{accuracy lower bound}, $Acc_{lower}$) do oceny dokładności tagerów,
\item miara ta karze wszelkie zmiany segmentacyjne w stosunku do złotego standardu i traktuje takie tokeny jako sklasyfikowane błędnie,
\item token traktowany jest jako oznakowany prawidłowo, jeśli zbiór jego interpretacji ma niepuste przecięcie ze zbiorem interpretacji zwracanych przez tager,
\item niezależne sprawdzamy dokładność dla znanych ($Acc^K_{lower}$) i nieznanych słów ($Acc^U_{lower}$), aby ocenić skuteczność ew. modułów odgadywania.
\end{itemize}
\end{frame}

\begin{frame}{Ewaluacja pojedynczych tagerów}
\structure{Eksperymenty na milionowym podkorpusie Narodowego Korpusu Języka Polskiego, ver. 1.1, 10-krotna walidacja krzyżowa.}
\begin{center}
\begin{tabular}{lcccc} \hline
n & Tager 		& $Acc_{lower}$	& $Acc^K_{lower}$	& $Acc^U_{lower}$	\\ \hline
1 & Pantera   & 88.95\%   & 91.22\% & 15.19\% \\
2 & WMBT	 	& 90.33\%		& 91.26\%	& 60.25\%	\\
3 & WCRFT	 	& 90.76\%		& 91.92\%	& 53.18\%	\\
4 & Concraft	& 91.07\%		& 92.06\%	& 58.81\%	\\
\end{tabular}
\end{center}
\begin{itemize}
\item $Acc_{lower}$ -- łączna dokładność,
\item $Acc^K_{lower}$ -- dokładność dla znanych słów,
\item $Acc^U_{lower}$ -- dokładność dla słów nieznanych.
\end{itemize}
\end{frame}

\begin{frame}{Analiza rezultatu działania tagerów}
\structure{Porównanie wyników}
\begin{itemize}
\item Wszystkie zwracają prawidłowy tag: \marked{82,78\%} \\
{\footnotesize \underline{unikam} fin:sg:pri:imperf\\
fin:sg:pri:imperf+ fin:sg:pri:imperf+ fin:sg:pri:imperf+ fin:sg:pri:imperf+}
\item Większość zwraca prawidłowy tag: \marked{7,95\%} \\
{\footnotesize \underline{kapitalistów} subst:pl:gen:m1 \\
subst:pl:gen:m1+ subst:pl:gen:m1+ subst:pl:gen:m1+ subst:pl:acc:m1-}
\item Równowaga w głosowaniu: \marked{2,71\%} \\
{\footnotesize \underline{powolny} adj:sg:nom:m3:pos \\
adj:sg:nom:m3:pos+ adj:sg:nom:m3:pos+ adj:sg:acc:m3:pos- adj:sg:acc:m3:pos-}
\item Prawidłowy tag w mniejszości: \marked{2,38\%} \\
{\footnotesize \underline{twarzy} subst:sg:loc:f subst:sg:gen:f- subst:sg:gen:f- subst:sg:gen:f- subst:sg:loc:f+}
\item Wszystkie się mylą: \marked{4.18\%} \\
{\footnotesize \underline{biurka} subst:pl:nom:n subst:pl:acc:n- subst:pl:acc:n- subst:sg:gen:n- subst:pl:acc:n- \\
(Peggy) \underline{McCreary} subst:sg:nom:f \\
subst:sg:gen:f- subst:sg:gen:n- subst:sg:nom:n- subst:sg:acc:m1-}
\end{itemize}
\end{frame}

\begin{frame}{Podział na klasy gramatyczne}
\begin{center}
\begin{tabular}{llllll}
 &  & \multicolumn{4}{c}{$Acc_{lower}$ (\%)} \\
klasa & liczność & PANTERA & WMBT & WCRFT & Concraft \\
\hline
subst & 331570 & 85,21 & 86,25 & 87,36 & 88,29 \\
interp & 223542 & 99,63 & 99,97 & 99,97 & 99,97 \\
adj & 128703 & 76,53 & 81,10 & 81,56 & 82,52 \\
prep & 115818 & 97,04 & 97,28 & 97,54 & 98,05 \\
qub & 68079 & 92,98 & 93,82 & 92,91 & 92,92 \\
fin & 59458 & 98,64 & 98,70 & 98,81 & 98,94 \\
praet & 53326 & 90,90 & 88,96 & 89,80 & 89,69 \\
conj & 44840 & 95,17 & 95,41 & 94,61 & 93,96 \\
adv & 42750 & 95,31 & 95,59 & 95,29 & 94,77 \\
inf & 19213 & 98,91 & 99,20 & 99,09 & 99,14 \\
comp & 17842 & 97,26 & 97,29 & 96,84 & 96,88 \\
num & 16160 & 33,40 & 56,40 & 60,32 & 55,99 \\
\end{tabular}
\end{center}
\end{frame}

% ile razy się zgadzają, równowaga, przewaga, różne
% w ile procentach są wszystkie poprawne, większość poprawna, mniejszość poprawna, żaden nie jest poprawny

\section{Tagery języka polskiego -- analiza jakościowa}
\frame{\sectionpage}

\section{Dyskusja i rekomendacje}
\frame{\sectionpage}

\begin{frame}{}
\begin{centering}
    \begin{beamercolorbox}[sep=12pt,center]{part title}
    \usebeamerfont{section title}Dziękujemy za uwagę!\par
    \end{beamercolorbox}
    \end{centering}
\end{frame}

\begin{frame}[allowframebreaks]
  \frametitle{Bibliografia}
  \begin{thebibliography}{10}
  \beamertemplatebookbibitems
  \bibitem[Acedański, 2010]{ace:2010}
Acedański, Szymon, 2010.
\newblock A morphosyntactic {B}rill tagger for inflectional languages.
\newblock In {\em Advances in Natural Language Processing\/}.
  \beamertemplatearticlebibitems
  \bibitem[Brill and Wu, 1998]{bri:wu:1998}
Brill, Eric and Jun Wu, 1998.
\newblock Classifier combination for improved lexical disambiguation.
\newblock In {\em Proceedings of the 17th international conference on
  Computational linguistics - Volume 1\/}, COLING '98. Stroudsburg, PA, USA:
  Association for Computational Linguistics.
  \beamertemplatearticlebibitems
  \bibitem[Śniatowski and Piasecki, 2012]{sni:pia:2012}
Śniatowski, Tomasz and Maciej Piasecki, 2012.
\newblock Combining {P}olish morphosyntactic taggers.
\newblock In Pascal Bouvry, Mieczysław~A. Kłopotek, Franck Leprévost,
  Małgorzata Marciniak, Agnieszka Mykowiecka, and Henryk Rybiński (eds.),
  {\em Security and Intelligent Information Systems\/}, volume 7053 of {\em
  LNCS\/}. Springer-Verlag.
  \beamertemplatearticlebibitems
  \bibitem[Radziszewski, 2013]{rad:2013}
Radziszewski, Adam, 2013.
\newblock A tiered {CRF} tagger for {P}olish.
\newblock In R.~Bembenik, {\L}.~Skonieczny, H.~Rybi\'{n}ski, M.~Kryszkiewicz,
  and M.~Niezg{\'o}dka (eds.), {\em Intelligent Tools for Building a Scientific
  Information Platform: Advanced Architectures and Solutions\/}. Springer
  Verlag.
  \beamertemplatearticlebibitems
  \bibitem[Radziszewski and Acedański, 2012]{rad:ace:2012}
Radziszewski, Adam and Szymon Acedański, 2012.
\newblock {T}aggers gonna tag: an argument against evaluating disambiguation
  capacities of morphosyntactic taggers.
\newblock In {\em Proceedings of TSD 2012\/}, LNCS. Springer-Verlag.
\beamertemplatearticlebibitems
\bibitem[Radziszewski and Śniatowski, 2011a]{rad:sni:2011}
Radziszewski, Adam and Tomasz Śniatowski, 2011a.
\newblock {A} {M}emory-{B}ased {T}agger for {P}olish.
\newblock In {\em Proceedings of the LTC 2011\/}.
\beamertemplatearticlebibitems
\bibitem[van Halteren et~al., 2001]{hal:dae:zav:2001}
van Halteren, Hans, Walter Daelemans, and Jakub Zavrel, 2001.
\newblock Improving accuracy in word class tagging through the combination of
  machine learning systems.
\newblock {\em Comput. Linguist.\/}, 27(2):199--229.
\beamertemplatearticlebibitems
\bibitem[Waszczuk, 2012]{wasz:12}
Waszczuk, Jakub, 2012.
\newblock Harnessing the {CRF} complexity with domain-specific constraints.
  {T}he case of morphosyntactic tagging of a highly inflected language.
\newblock In {\em Proceedings of the 24th International Conference on
  Computational Linguistics ({COLING}\,2012)\/}. Mumbai, India.
  \end{thebibliography}
\end{frame}

\end{document}
